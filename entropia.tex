\documentclass{article}
\usepackage[utf8]{inputenc}
\usepackage{indentfirst}
\setlength{\parindent}{1cm}
\usepackage{graphicx}

\begin{document}

\begin{titlepage}

    \centering
    \vspace*{4,8cm}
    \Huge \textbf{Entropia de Dados}

    \vspace{1cm}
    \Large 1º Ciclo - Ciência de Dados
    
    \vspace{0.5cm}
    \Large Ana Luiza Ribeiro de Santana da Silva
    
    \Large Diego Conceição Santos de Faria

    \Large Rebeca Nascimento Oliveira 
    
    \vspace{1cm}
    \Large Junho, 2023
    

\end{titlepage}

\section*{Base de dados}

A base de dados utilizada foi adquirida a partir de um conjunto de dados proveniente da University of California - Irvine, uma instituição de ensino superior localizada nos Estados Unidos. Esta base de dados consiste em informações qualitativas e foi obtida diretamente do site UCI Machine Learning Repository.

A base de dados compreendia as medidas de comprimento e largura da sépala em centímetros, bem como as medidas de comprimento e largura da pétala em centímetros. Com base nas dimensões da largura e do comprimento, a classificação das amostras podia ser atribuída a uma das três classes: Iris Setosa, Iris Versicolour ou Iris Virginica.

\section*{Resultados obtidos}
As classes presentes no conjunto de dados são as seguintes:

\begin{itemize}
    \item Iris Setosa;
    \item Iris Versicolour;
    \item Iris Virginica.
\end{itemize}

\includegraphics[width = 10cm]{flores de íris.png}

\vspace{0,5cm}

O dataset é composto por 150 instâncias distribuídas igualmente em 3 classes, onde cada classe representa um tipo de planta de íris. Nesse contexto, uma das classes é linearmente separável das outras duas, o que implica que é possível traçar uma linha ou plano no espaço de atributos que permite separar essa classe das demais. No entanto, as duas últimas classes não são linearmente separáveis entre si, o que significa que não é possível encontrar uma linha ou plano que as separe de forma linear. Isso indica que essas duas classes apresentam sobreposição em seus padrões de atributos e não podem ser claramente distinguidas por meio de uma separação linear. Portanto, é necessário utilizar métodos de classificação mais complexos para uma distinção precisa entre essas classes.

\vspace{0,5cm}

A entropia foi calculada por intermédio desse código:

\begin{verbatim}
import numpy as np
import pandas as pd
import math

dados = pd.read_csv('D:\Downlods\iris.csv', delimiter=',')
coluna_qualitativa = dados['class']
frequencia = coluna_qualitativa.value_counts()
proporcao = frequencia / len(coluna_qualitativa)
entropia = -sum(proporcao * np.log2(proporcao))

print('Entropia: ', entropia)
\end{verbatim}

Para realizar o cálculo da entropia máxima, o seguinte código foi utilizado:

\begin{verbatim}
import math

def calcular_entropia_maxima(base_dados):
    total_registros = len(base_dados)
    valores_unicos = set(base_dados)
    num_valores_unicos = len(valores_unicos)
    probabilidade = 1 / num_valores_unicos
    entropia_maxima = -1 * sum(probabilidade * math.log2(probabilidade) 
    for n in range(num_valores_unicos))
    return entropia_maxima

dados = 'D:\Downlods\iris.csv'
entropia_max = calcular_entropia_maxima(dados)
print(f"A entropia máxima da base de dados é: {entropia_max}")
\end{verbatim}

\section*{Conclusão}
\begin{itemize}
    \item Resultado da entropia: 1.584962500721156
    \item Resultado da entropia máxima: 3.8073549220576055
\end{itemize}

\includegraphics[width = 11cm]{console.png}

\vspace{0,5cm}

Ao analisarmos os valores obtidos, é possível observar que os dados apresentam um nível moderado de aleatoriedade.

\end{document}
